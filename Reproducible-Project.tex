% Options for packages loaded elsewhere
\PassOptionsToPackage{unicode}{hyperref}
\PassOptionsToPackage{hyphens}{url}
%
\documentclass[
]{article}
\usepackage{lmodern}
\usepackage{amssymb,amsmath}
\usepackage{ifxetex,ifluatex}
\ifnum 0\ifxetex 1\fi\ifluatex 1\fi=0 % if pdftex
  \usepackage[T1]{fontenc}
  \usepackage[utf8]{inputenc}
  \usepackage{textcomp} % provide euro and other symbols
\else % if luatex or xetex
  \usepackage{unicode-math}
  \defaultfontfeatures{Scale=MatchLowercase}
  \defaultfontfeatures[\rmfamily]{Ligatures=TeX,Scale=1}
\fi
% Use upquote if available, for straight quotes in verbatim environments
\IfFileExists{upquote.sty}{\usepackage{upquote}}{}
\IfFileExists{microtype.sty}{% use microtype if available
  \usepackage[]{microtype}
  \UseMicrotypeSet[protrusion]{basicmath} % disable protrusion for tt fonts
}{}
\makeatletter
\@ifundefined{KOMAClassName}{% if non-KOMA class
  \IfFileExists{parskip.sty}{%
    \usepackage{parskip}
  }{% else
    \setlength{\parindent}{0pt}
    \setlength{\parskip}{6pt plus 2pt minus 1pt}}
}{% if KOMA class
  \KOMAoptions{parskip=half}}
\makeatother
\usepackage{xcolor}
\IfFileExists{xurl.sty}{\usepackage{xurl}}{} % add URL line breaks if available
\IfFileExists{bookmark.sty}{\usepackage{bookmark}}{\usepackage{hyperref}}
\hypersetup{
  pdftitle={Reproducible Project},
  pdfauthor={H. Tetteh},
  hidelinks,
  pdfcreator={LaTeX via pandoc}}
\urlstyle{same} % disable monospaced font for URLs
\usepackage[margin=1in]{geometry}
\usepackage{color}
\usepackage{fancyvrb}
\newcommand{\VerbBar}{|}
\newcommand{\VERB}{\Verb[commandchars=\\\{\}]}
\DefineVerbatimEnvironment{Highlighting}{Verbatim}{commandchars=\\\{\}}
% Add ',fontsize=\small' for more characters per line
\usepackage{framed}
\definecolor{shadecolor}{RGB}{248,248,248}
\newenvironment{Shaded}{\begin{snugshade}}{\end{snugshade}}
\newcommand{\AlertTok}[1]{\textcolor[rgb]{0.94,0.16,0.16}{#1}}
\newcommand{\AnnotationTok}[1]{\textcolor[rgb]{0.56,0.35,0.01}{\textbf{\textit{#1}}}}
\newcommand{\AttributeTok}[1]{\textcolor[rgb]{0.77,0.63,0.00}{#1}}
\newcommand{\BaseNTok}[1]{\textcolor[rgb]{0.00,0.00,0.81}{#1}}
\newcommand{\BuiltInTok}[1]{#1}
\newcommand{\CharTok}[1]{\textcolor[rgb]{0.31,0.60,0.02}{#1}}
\newcommand{\CommentTok}[1]{\textcolor[rgb]{0.56,0.35,0.01}{\textit{#1}}}
\newcommand{\CommentVarTok}[1]{\textcolor[rgb]{0.56,0.35,0.01}{\textbf{\textit{#1}}}}
\newcommand{\ConstantTok}[1]{\textcolor[rgb]{0.00,0.00,0.00}{#1}}
\newcommand{\ControlFlowTok}[1]{\textcolor[rgb]{0.13,0.29,0.53}{\textbf{#1}}}
\newcommand{\DataTypeTok}[1]{\textcolor[rgb]{0.13,0.29,0.53}{#1}}
\newcommand{\DecValTok}[1]{\textcolor[rgb]{0.00,0.00,0.81}{#1}}
\newcommand{\DocumentationTok}[1]{\textcolor[rgb]{0.56,0.35,0.01}{\textbf{\textit{#1}}}}
\newcommand{\ErrorTok}[1]{\textcolor[rgb]{0.64,0.00,0.00}{\textbf{#1}}}
\newcommand{\ExtensionTok}[1]{#1}
\newcommand{\FloatTok}[1]{\textcolor[rgb]{0.00,0.00,0.81}{#1}}
\newcommand{\FunctionTok}[1]{\textcolor[rgb]{0.00,0.00,0.00}{#1}}
\newcommand{\ImportTok}[1]{#1}
\newcommand{\InformationTok}[1]{\textcolor[rgb]{0.56,0.35,0.01}{\textbf{\textit{#1}}}}
\newcommand{\KeywordTok}[1]{\textcolor[rgb]{0.13,0.29,0.53}{\textbf{#1}}}
\newcommand{\NormalTok}[1]{#1}
\newcommand{\OperatorTok}[1]{\textcolor[rgb]{0.81,0.36,0.00}{\textbf{#1}}}
\newcommand{\OtherTok}[1]{\textcolor[rgb]{0.56,0.35,0.01}{#1}}
\newcommand{\PreprocessorTok}[1]{\textcolor[rgb]{0.56,0.35,0.01}{\textit{#1}}}
\newcommand{\RegionMarkerTok}[1]{#1}
\newcommand{\SpecialCharTok}[1]{\textcolor[rgb]{0.00,0.00,0.00}{#1}}
\newcommand{\SpecialStringTok}[1]{\textcolor[rgb]{0.31,0.60,0.02}{#1}}
\newcommand{\StringTok}[1]{\textcolor[rgb]{0.31,0.60,0.02}{#1}}
\newcommand{\VariableTok}[1]{\textcolor[rgb]{0.00,0.00,0.00}{#1}}
\newcommand{\VerbatimStringTok}[1]{\textcolor[rgb]{0.31,0.60,0.02}{#1}}
\newcommand{\WarningTok}[1]{\textcolor[rgb]{0.56,0.35,0.01}{\textbf{\textit{#1}}}}
\usepackage{graphicx,grffile}
\makeatletter
\def\maxwidth{\ifdim\Gin@nat@width>\linewidth\linewidth\else\Gin@nat@width\fi}
\def\maxheight{\ifdim\Gin@nat@height>\textheight\textheight\else\Gin@nat@height\fi}
\makeatother
% Scale images if necessary, so that they will not overflow the page
% margins by default, and it is still possible to overwrite the defaults
% using explicit options in \includegraphics[width, height, ...]{}
\setkeys{Gin}{width=\maxwidth,height=\maxheight,keepaspectratio}
% Set default figure placement to htbp
\makeatletter
\def\fps@figure{htbp}
\makeatother
\setlength{\emergencystretch}{3em} % prevent overfull lines
\providecommand{\tightlist}{%
  \setlength{\itemsep}{0pt}\setlength{\parskip}{0pt}}
\setcounter{secnumdepth}{-\maxdimen} % remove section numbering

\title{Reproducible Project}
\author{H. Tetteh}
\date{21/03/2020}

\begin{document}
\maketitle

This assignment will be described in multiple parts. You will need to
write a report that answers the questions detailed below. Ultimately,
you will need to complete the entire assignment in a single R markdown
document that can be processed by knitr and be transformed into an HTML
file.

Throughout your report make sure you always include the code that you
used to generate the output you present. When writing code chunks in the
R markdown document, always use echo = TRUE so that someone else will be
able to read the code. This assignment will be evaluated via peer
assessment so it is essential that your peer evaluators be able to
review the code for your analysis.

For the plotting aspects of this assignment, feel free to use any
plotting system in R (i.e., base, lattice, ggplot2)

Fork/clone the GitHub repository created for this assignment. You will
submit this assignment by pushing your completed files into your forked
repository on GitHub. The assignment submission will consist of the URL
to your GitHub repository and the SHA-1 commit ID for your repository
state.

NOTE: The GitHub repository also contains the dataset for the assignment
so you do not have to download the data separately.

\hypertarget{loading-and-preprocessing-the-data}{%
\subsection{1) Loading and preprocessing the
data}\label{loading-and-preprocessing-the-data}}

\begin{Shaded}
\begin{Highlighting}[]
\ControlFlowTok{if}\NormalTok{ (}\OperatorTok{!}\KeywordTok{file.exists}\NormalTok{(}\StringTok{'activity.csv'}\NormalTok{))\{}
  \KeywordTok{unzip}\NormalTok{(}\DataTypeTok{zipfile =} \StringTok{"activity.zip"}\NormalTok{)}
\NormalTok{\}}

\NormalTok{activityDataSet <-}\StringTok{ }\KeywordTok{read.csv}\NormalTok{( }\DataTypeTok{file =} \StringTok{"activity.csv"}\NormalTok{, }\DataTypeTok{header =} \OtherTok{TRUE}\NormalTok{)}
 
 \KeywordTok{head}\NormalTok{(activityDataSet, }\DecValTok{20}\NormalTok{)}
\end{Highlighting}
\end{Shaded}

\begin{verbatim}
##    steps       date interval
## 1     NA 2012-10-01        0
## 2     NA 2012-10-01        5
## 3     NA 2012-10-01       10
## 4     NA 2012-10-01       15
## 5     NA 2012-10-01       20
## 6     NA 2012-10-01       25
## 7     NA 2012-10-01       30
## 8     NA 2012-10-01       35
## 9     NA 2012-10-01       40
## 10    NA 2012-10-01       45
## 11    NA 2012-10-01       50
## 12    NA 2012-10-01       55
## 13    NA 2012-10-01      100
## 14    NA 2012-10-01      105
## 15    NA 2012-10-01      110
## 16    NA 2012-10-01      115
## 17    NA 2012-10-01      120
## 18    NA 2012-10-01      125
## 19    NA 2012-10-01      130
## 20    NA 2012-10-01      135
\end{verbatim}

\begin{Shaded}
\begin{Highlighting}[]
 \KeywordTok{tail}\NormalTok{(activityDataSet, }\DecValTok{20}\NormalTok{)}
\end{Highlighting}
\end{Shaded}

\begin{verbatim}
##       steps       date interval
## 17549    NA 2012-11-30     2220
## 17550    NA 2012-11-30     2225
## 17551    NA 2012-11-30     2230
## 17552    NA 2012-11-30     2235
## 17553    NA 2012-11-30     2240
## 17554    NA 2012-11-30     2245
## 17555    NA 2012-11-30     2250
## 17556    NA 2012-11-30     2255
## 17557    NA 2012-11-30     2300
## 17558    NA 2012-11-30     2305
## 17559    NA 2012-11-30     2310
## 17560    NA 2012-11-30     2315
## 17561    NA 2012-11-30     2320
## 17562    NA 2012-11-30     2325
## 17563    NA 2012-11-30     2330
## 17564    NA 2012-11-30     2335
## 17565    NA 2012-11-30     2340
## 17566    NA 2012-11-30     2345
## 17567    NA 2012-11-30     2350
## 17568    NA 2012-11-30     2355
\end{verbatim}

\hypertarget{what-is-mean-total-number-of-steps-taken-per-day}{%
\subsection{2) What is mean total number of steps taken per
day?}\label{what-is-mean-total-number-of-steps-taken-per-day}}

\begin{Shaded}
\begin{Highlighting}[]
\CommentTok{# a) Make a histogram of the total number of steps taken each day}
\NormalTok{totalSteps <-}\StringTok{ }\KeywordTok{aggregate}\NormalTok{(steps }\OperatorTok{~}\StringTok{ }\NormalTok{date, activityDataSet, }\DataTypeTok{FUN=}\NormalTok{sum)}

\KeywordTok{hist}\NormalTok{(totalSteps}\OperatorTok{$}\NormalTok{steps,}
     \DataTypeTok{main =} \StringTok{"Total Steps per Day"}\NormalTok{,}
     \DataTypeTok{xlab =} \StringTok{"Number of Steps"}\NormalTok{)}
\end{Highlighting}
\end{Shaded}

\includegraphics{Reproducible-Project_files/figure-latex/Calculate the total steps taken per day-1.pdf}

\begin{Shaded}
\begin{Highlighting}[]
\CommentTok{# b) Calculate and report the mean and median of total steps taken per day *Excluding NA*}
\CommentTok{#Total number of NAs}

\KeywordTok{sum}\NormalTok{(}\KeywordTok{is.na}\NormalTok{(activityDataSet))}
\end{Highlighting}
\end{Shaded}

\begin{verbatim}
## [1] 2304
\end{verbatim}

\begin{Shaded}
\begin{Highlighting}[]
\NormalTok{mean_Steps_per_day <-}\StringTok{ }\KeywordTok{mean}\NormalTok{(totalSteps}\OperatorTok{$}\NormalTok{steps, }\DataTypeTok{na.rm =} \OtherTok{TRUE}\NormalTok{)}
\NormalTok{mean_Steps_per_day}
\end{Highlighting}
\end{Shaded}

\begin{verbatim}
## [1] 10766.19
\end{verbatim}

\begin{Shaded}
\begin{Highlighting}[]
\NormalTok{median_Steps_day <-}\StringTok{ }\KeywordTok{median}\NormalTok{(totalSteps}\OperatorTok{$}\NormalTok{steps, }\DataTypeTok{na.rm =} \OtherTok{TRUE}\NormalTok{)}
\NormalTok{median_Steps_day}
\end{Highlighting}
\end{Shaded}

\begin{verbatim}
## [1] 10765
\end{verbatim}

\hypertarget{what-is-the-average-daily-activity-pattern}{%
\subsection{3) What is the average daily activity
pattern?}\label{what-is-the-average-daily-activity-pattern}}

\begin{Shaded}
\begin{Highlighting}[]
\CommentTok{# preprocessing data for plot}
\NormalTok{Steps_by_interval <-}\StringTok{ }\KeywordTok{aggregate}\NormalTok{(steps }\OperatorTok{~}\StringTok{ }\NormalTok{interval, activityDataSet, mean)}

\CommentTok{# create a time series plot }
\KeywordTok{plot}\NormalTok{(Steps_by_interval}\OperatorTok{$}\NormalTok{interval, Steps_by_interval}\OperatorTok{$}\NormalTok{steps, }\DataTypeTok{type=}\StringTok{'l'}\NormalTok{, }
     \DataTypeTok{main=}\StringTok{"Average number of steps over all days"}\NormalTok{, }\DataTypeTok{xlab=}\StringTok{"5 - Minute Interval"}\NormalTok{, }
     \DataTypeTok{ylab=}\StringTok{"Average number of steps"}\NormalTok{)}
\end{Highlighting}
\end{Shaded}

\includegraphics{Reproducible-Project_files/figure-latex/What is the average daily activity pattern-1.pdf}

\hypertarget{the-5-minute-interval-on-average-across-all-the-days-in-the-dataset-containing-the-maximum-number-of-steps.}{%
\subsubsection{The 5-minute interval, on average across all the days in
the dataset, containing the maximum number of
steps.}\label{the-5-minute-interval-on-average-across-all-the-days-in-the-dataset-containing-the-maximum-number-of-steps.}}

\begin{Shaded}
\begin{Highlighting}[]
\CommentTok{# find row with max of steps}
\NormalTok{max_steps_row <-}\StringTok{ }\KeywordTok{which.max}\NormalTok{(Steps_by_interval}\OperatorTok{$}\NormalTok{steps)}

\CommentTok{# find interval with this max}
\NormalTok{Steps_by_interval[max_steps_row,]}
\end{Highlighting}
\end{Shaded}

\begin{verbatim}
##     interval    steps
## 104      835 206.1698
\end{verbatim}

\hypertarget{the-interval-835-has-the-maximum-average-value-of-steps-206.1698.}{%
\subsubsection{The interval 835 has the maximum average value of steps
(206.1698).}\label{the-interval-835-has-the-maximum-average-value-of-steps-206.1698.}}

\hypertarget{imputing-missing-values}{%
\subsection{4 Imputing missing values}\label{imputing-missing-values}}

\hypertarget{a-calculate-and-report-the-total-number-of-missing-values}{%
\subsubsection{a) Calculate and report the total number of missing
values}\label{a-calculate-and-report-the-total-number-of-missing-values}}

\begin{Shaded}
\begin{Highlighting}[]
\KeywordTok{sum}\NormalTok{(}\KeywordTok{is.na}\NormalTok{(activityDataSet))}
\end{Highlighting}
\end{Shaded}

\begin{verbatim}
## [1] 2304
\end{verbatim}

\#\#\#Total rows with missing Values 2304 Rows

\hypertarget{bi-replace-every-nas-with-the-mean-for-that-5-minute-intervalsince-2304-is-a-large-number-i-can-not-remove-affected-rows.}{%
\subsection{b)I replace every NA's with the mean for that 5-minute
interval,since 2304 is a large number i can not remove affected
rows.}\label{bi-replace-every-nas-with-the-mean-for-that-5-minute-intervalsince-2304-is-a-large-number-i-can-not-remove-affected-rows.}}

\begin{Shaded}
\begin{Highlighting}[]
\NormalTok{activityDataSet_imputed <-}\StringTok{ }\NormalTok{activityDataSet}
\ControlFlowTok{for}\NormalTok{ (i }\ControlFlowTok{in} \DecValTok{1}\OperatorTok{:}\KeywordTok{nrow}\NormalTok{(activityDataSet_imputed)) \{}
  \ControlFlowTok{if}\NormalTok{ (}\KeywordTok{is.na}\NormalTok{(activityDataSet_imputed}\OperatorTok{$}\NormalTok{steps[i])) \{}
\NormalTok{    interval_value <-}\StringTok{ }\NormalTok{activityDataSet_imputed}\OperatorTok{$}\NormalTok{interval[i]}
\NormalTok{    steps_value <-}\StringTok{ }\NormalTok{Steps_by_interval[}
\NormalTok{     Steps_by_interval}\OperatorTok{$}\NormalTok{interval }\OperatorTok{==}\StringTok{ }\NormalTok{interval_value,]}
\NormalTok{    activityDataSet_imputed}\OperatorTok{$}\NormalTok{steps[i] <-}\StringTok{ }\NormalTok{steps_value}\OperatorTok{$}\NormalTok{steps}
\NormalTok{  \}}
\NormalTok{\}}
\end{Highlighting}
\end{Shaded}

\hypertarget{ive-created-new-data-set-data_no_na-which-equals-to-activitydataset_imputed_row-but-without-nas.-all-nas-are-replaced-with-mean-of-5-minute-interval.}{%
\subsection{I've created new data set data\_no\_na which equals to
activityDataSet\_imputed\_row but without NA's. All NA's are replaced
with mean of 5-minute
interval.}\label{ive-created-new-data-set-data_no_na-which-equals-to-activitydataset_imputed_row-but-without-nas.-all-nas-are-replaced-with-mean-of-5-minute-interval.}}

\hypertarget{c-printing-the-first-and-last-20-rows-of-the-new-datasetactivitydataset_imputed}{%
\subsection{c) printing the first and last 20 rows of the new
Dataset(activityDataSet\_imputed)}\label{c-printing-the-first-and-last-20-rows-of-the-new-datasetactivitydataset_imputed}}

\begin{Shaded}
\begin{Highlighting}[]
\CommentTok{#activityDataSet_imputed}
\KeywordTok{head}\NormalTok{(activityDataSet_imputed, }\DecValTok{20}\NormalTok{)}
\end{Highlighting}
\end{Shaded}

\begin{verbatim}
##        steps       date interval
## 1  1.7169811 2012-10-01        0
## 2  0.3396226 2012-10-01        5
## 3  0.1320755 2012-10-01       10
## 4  0.1509434 2012-10-01       15
## 5  0.0754717 2012-10-01       20
## 6  2.0943396 2012-10-01       25
## 7  0.5283019 2012-10-01       30
## 8  0.8679245 2012-10-01       35
## 9  0.0000000 2012-10-01       40
## 10 1.4716981 2012-10-01       45
## 11 0.3018868 2012-10-01       50
## 12 0.1320755 2012-10-01       55
## 13 0.3207547 2012-10-01      100
## 14 0.6792453 2012-10-01      105
## 15 0.1509434 2012-10-01      110
## 16 0.3396226 2012-10-01      115
## 17 0.0000000 2012-10-01      120
## 18 1.1132075 2012-10-01      125
## 19 1.8301887 2012-10-01      130
## 20 0.1698113 2012-10-01      135
\end{verbatim}

\begin{Shaded}
\begin{Highlighting}[]
\KeywordTok{tail}\NormalTok{(activityDataSet_imputed, }\DecValTok{20}\NormalTok{)}
\end{Highlighting}
\end{Shaded}

\begin{verbatim}
##           steps       date interval
## 17549 7.0754717 2012-11-30     2220
## 17550 8.6981132 2012-11-30     2225
## 17551 9.7547170 2012-11-30     2230
## 17552 2.2075472 2012-11-30     2235
## 17553 0.3207547 2012-11-30     2240
## 17554 0.1132075 2012-11-30     2245
## 17555 1.6037736 2012-11-30     2250
## 17556 4.6037736 2012-11-30     2255
## 17557 3.3018868 2012-11-30     2300
## 17558 2.8490566 2012-11-30     2305
## 17559 0.0000000 2012-11-30     2310
## 17560 0.8301887 2012-11-30     2315
## 17561 0.9622642 2012-11-30     2320
## 17562 1.5849057 2012-11-30     2325
## 17563 2.6037736 2012-11-30     2330
## 17564 4.6981132 2012-11-30     2335
## 17565 3.3018868 2012-11-30     2340
## 17566 0.6415094 2012-11-30     2345
## 17567 0.2264151 2012-11-30     2350
## 17568 1.0754717 2012-11-30     2355
\end{verbatim}

\hypertarget{checking-how-many-rows-of-nas-are-affected}{%
\subsection{checking how many Rows of NAs are
affected}\label{checking-how-many-rows-of-nas-are-affected}}

\begin{Shaded}
\begin{Highlighting}[]
\KeywordTok{sum}\NormalTok{(}\KeywordTok{is.na}\NormalTok{(activityDataSet_imputed))}
\end{Highlighting}
\end{Shaded}

\begin{verbatim}
## [1] 0
\end{verbatim}

\hypertarget{ive-created-new-data-set-data_no_na-which-equals-to-data_row-but-without-nas.-all-nas-are-replaced-with-mean-of-5-minute-interval-so-na-is-equal-to-0-now.}{%
\subsubsection{I've created new data set data\_no\_NA which equals to
data\_row but without NA's. All NA's are replaced with mean of 5-minute
interval, so NA is equal to 0
now.}\label{ive-created-new-data-set-data_no_na-which-equals-to-data_row-but-without-nas.-all-nas-are-replaced-with-mean-of-5-minute-interval-so-na-is-equal-to-0-now.}}

\hypertarget{d-make-a-histogram-of-the-total-number-of-steps-taken-each-day-and-calculate-and-report-the-mean-and-median-total-number-of-steps-taken-per-day}{%
\subsection{d) Make a histogram of the total number of steps taken each
day and Calculate and report the mean and median total number of steps
taken per
day}\label{d-make-a-histogram-of-the-total-number-of-steps-taken-each-day-and-calculate-and-report-the-mean-and-median-total-number-of-steps-taken-per-day}}

\begin{Shaded}
\begin{Highlighting}[]
\NormalTok{df_imputed_steps <-}\StringTok{ }\KeywordTok{aggregate}\NormalTok{(steps }\OperatorTok{~}\StringTok{ }\NormalTok{date, activityDataSet_imputed , sum)}
\KeywordTok{head}\NormalTok{(df_imputed_steps, }\DecValTok{20}\NormalTok{)}
\end{Highlighting}
\end{Shaded}

\begin{verbatim}
##          date    steps
## 1  2012-10-01 10766.19
## 2  2012-10-02   126.00
## 3  2012-10-03 11352.00
## 4  2012-10-04 12116.00
## 5  2012-10-05 13294.00
## 6  2012-10-06 15420.00
## 7  2012-10-07 11015.00
## 8  2012-10-08 10766.19
## 9  2012-10-09 12811.00
## 10 2012-10-10  9900.00
## 11 2012-10-11 10304.00
## 12 2012-10-12 17382.00
## 13 2012-10-13 12426.00
## 14 2012-10-14 15098.00
## 15 2012-10-15 10139.00
## 16 2012-10-16 15084.00
## 17 2012-10-17 13452.00
## 18 2012-10-18 10056.00
## 19 2012-10-19 11829.00
## 20 2012-10-20 10395.00
\end{verbatim}

\begin{Shaded}
\begin{Highlighting}[]
\KeywordTok{sum}\NormalTok{(}\KeywordTok{is.na}\NormalTok{(df_imputed_steps))}
\end{Highlighting}
\end{Shaded}

\begin{verbatim}
## [1] 0
\end{verbatim}

\hypertarget{calculating-the-mean-and-median-of-imputed-data}{%
\subsubsection{Calculating the mean and median of imputed
data}\label{calculating-the-mean-and-median-of-imputed-data}}

\begin{Shaded}
\begin{Highlighting}[]
\KeywordTok{mean}\NormalTok{(df_imputed_steps}\OperatorTok{$}\NormalTok{steps)}
\end{Highlighting}
\end{Shaded}

\begin{verbatim}
## [1] 10766.19
\end{verbatim}

\begin{Shaded}
\begin{Highlighting}[]
\KeywordTok{median}\NormalTok{(df_imputed_steps}\OperatorTok{$}\NormalTok{steps)}
\end{Highlighting}
\end{Shaded}

\begin{verbatim}
## [1] 10766.19
\end{verbatim}

\begin{Shaded}
\begin{Highlighting}[]
\KeywordTok{hist}\NormalTok{(df_imputed_steps}\OperatorTok{$}\NormalTok{steps, }\DataTypeTok{main=}\StringTok{"Histogram of total number of steps per day (imputed)"}\NormalTok{, }
\DataTypeTok{xlab=}\StringTok{"Total number of steps in a day"}\NormalTok{)}
\end{Highlighting}
\end{Shaded}

\includegraphics{Reproducible-Project_files/figure-latex/Make a histogram of the total number of steps taken each day-1.pdf}

\hypertarget{are-there-differences-in-activity-patterns-between-weekdays-and-weekends}{%
\subsection{5) Are there differences in activity patterns between
weekdays and
weekends?}\label{are-there-differences-in-activity-patterns-between-weekdays-and-weekends}}

\begin{Shaded}
\begin{Highlighting}[]
\NormalTok{activityDataSet_imputed[}\StringTok{'type_of_day'}\NormalTok{] <-}\StringTok{ }\KeywordTok{weekdays}\NormalTok{(}\KeywordTok{as.Date}\NormalTok{(activityDataSet_imputed}\OperatorTok{$}\NormalTok{date))}
\NormalTok{activityDataSet_imputed}\OperatorTok{$}\NormalTok{type_of_day[activityDataSet_imputed}\OperatorTok{$}\NormalTok{type_of_day  }\OperatorTok\StringTok{ }\KeywordTok{c}\NormalTok{(}\StringTok{'Saturday'}\NormalTok{,}\StringTok{'Sunday'}\NormalTok{) ] <-}\StringTok{ "weekend"}
\NormalTok{activityDataSet_imputed}\OperatorTok{$}\NormalTok{type_of_day[activityDataSet_imputed}\OperatorTok{$}\NormalTok{type_of_day }\OperatorTok{!=}\StringTok{ "weekend"}\NormalTok{] <-}\StringTok{ "weekday"}
\end{Highlighting}
\end{Shaded}

\begin{Shaded}
\begin{Highlighting}[]
\NormalTok{activityDataSet_imputed}\OperatorTok{$}\NormalTok{type_of_day <-}\StringTok{ }\KeywordTok{as.factor}\NormalTok{(activityDataSet_imputed}\OperatorTok{$}\NormalTok{type_of_day)}
\end{Highlighting}
\end{Shaded}

\begin{Shaded}
\begin{Highlighting}[]
\NormalTok{df_imputed_steps_by_interval <-}\StringTok{ }\KeywordTok{aggregate}\NormalTok{(steps }\OperatorTok{~}\StringTok{ }\NormalTok{interval }\OperatorTok{+}\StringTok{ }\NormalTok{type_of_day, activityDataSet_imputed , mean)}
\end{Highlighting}
\end{Shaded}

\begin{Shaded}
\begin{Highlighting}[]
\KeywordTok{qplot}\NormalTok{(interval, }
\NormalTok{      steps, }
      \DataTypeTok{data =}\NormalTok{ df_imputed_steps_by_interval, }
      \DataTypeTok{type =} \StringTok{'l'}\NormalTok{, }
      \DataTypeTok{geom=}\KeywordTok{c}\NormalTok{(}\StringTok{"line"}\NormalTok{),}
      \DataTypeTok{xlab =} \StringTok{"Interval"}\NormalTok{, }
      \DataTypeTok{ylab =} \StringTok{"Number of steps"}\NormalTok{, }
      \DataTypeTok{main =} \StringTok{"Average Daily Activity Pattern"}\NormalTok{) }\OperatorTok{+}
\StringTok{  }\KeywordTok{facet_wrap}\NormalTok{(}\OperatorTok{~}\StringTok{ }\NormalTok{type_of_day, }\DataTypeTok{ncol =} \DecValTok{1}\NormalTok{)}
\end{Highlighting}
\end{Shaded}

\begin{verbatim}
## Warning: Ignoring unknown parameters: type
\end{verbatim}

\includegraphics{Reproducible-Project_files/figure-latex/The Make a panel plot containing a time series plot-1.pdf}

\end{document}
